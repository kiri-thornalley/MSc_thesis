\chapter{Summary and Conclusions} % Main chapter title
\label{Chapter3} %

\section{Synthesis of novel nanostructures}
Two enantiomeric pairs of target compounds have been successfully synthesised via a synthetic method  developed in this work, and their identity confirmed by \textsuperscript{1}H, \textsuperscript{13}C and DEPT-135 NMR, Mass Spectrometry and Infrared Spectroscopy.  It has been noted that modifying the amino acid spacer group has a substantial effect on the solubility of some intermediate products. This was unexpected, but from previous work, it is not surprising that a small change in the structure of these types of molecules leads to a large change in their behaviour.\textsuperscript{\cite{Vieira2017EmergenceHeparin,Albanyan2017Self-AssembledLigands}}  

There are a variety of possible ways to remove the Boc-protecting groups from these products. Both dissolving the Boc-protected product in DCM before bubbling HCl gas through the solution and dissolving the product in 4M HCl in dioxane have been used and both successfully remove the protecting group, as shown by \textsuperscript{1}H NMR. 

\textsuperscript{1}H NMR has also been used to show that there are subtle differences between the four novel heparin binders in this work. Key proton resonances all appear at equivalent ppm values. However, the coupling patterns of the CH\textsubscript{2}-N protons differ between the two diastereotopic pairs of compounds, allowing them to be distinguished. Also proving that the chirality of the amino acid building blocks does not get scrambled during synthesis, and suggests that the diastereomers may have different conformational preferences. 

\section{Assessing Self-Assembly}
\subsection{Nile Red Assay}
The Nile Red assays show that all four novel heparin binders successfully aggregate in solution, and the calculated CAC values for the C\textsubscript{16}-L-Ala-L-Lys, C\textsubscript{16}-D-Ala-D-Lys pair of molecules are similar to those obtained by Vieira and Chan for C\textsubscript{16}-DAPMA and C\textsubscript{16}-Gly-L-Lys respectively.\textsuperscript{\cite{Vieira2017EmergenceHeparin,Chan2016ChiralBinding}}
It also shows that the insertion of the chiral amino acid alanine does have an effect on the properties of the overall nanoscale assembly, as the CAC value increases significantly from 45 \textmu M for C\textsubscript{16}-L-Ala-L-Lys/C\textsubscript{16}-D-Ala-D-Lys to 160 \textmu M for the diastereomeric pair C\textsubscript{16}-D-Ala-L-Lys and C\textsubscript{16}-L-Lys-D-Ala. This dramatic increase in CAC value was not observed by Chan \textit{et al.} on the insertion of the achiral amino acid glysine into C\textsubscript{16}-Lys, but changing chirality leading to significant changes in behaviour has been reported previously by Boutellier \textit{et al.} for a 1,3,5-benzene-tricarboxamide system.\textsuperscript{\cite{Chan2016ChiralBinding,Caumes2016TuningStereochemistry}} It suggests that the conformational change observed by \textsuperscript{1}H NMR significantly impacts on the ability of these compounds to self-assemble. It is hypothesised that the head group becomes more sterically hindered and less able to self-assemble effectively. 

\subsection{Dynamic Light Scattering}
DLS can be used to obtain a large variety of information about a sample, however, one of the main assumptions made by DLS is that the objects which scatter light are spherical. Hence this analytical method can say very little about the shape of the self-assembled heparin binders in solution. Despite this, the polydispersity index (PDI), Intensity and Volume distributions hold useful information. 

The PDI of these samples is between 0.5 and 0.6, indicating that these samples are not monodisperse (PDI = 1.0) and that a range of different sized objects must be present within the sample. This implies the presence of aggregates within the solution. 

In general terms, the intensity distributions show two distinct peaks; one at smaller size (<10 nm) that corresponds to the micellar objects, and a second at a substantially larger size (100-500 nm), indicating the presence of aggregates within the sample. The volume distribution indicates that the dominant species in each sample are these smaller micellar-type assemblies.  

It can also be seen that the systems which self-assemble more effectively also broadly appear to posses more reproducible DLS traces, which helps support the hypothesis that they form better defined aggregates in comparison to those formed by C\textsubscript{16}-D-Ala-L-Lys and C\textsubscript{16}-L-Ala-D-Lys. It should also be noted that some of these larger aggregates that have been observed are an artifact of the assay being performed at micromolar concentrations - this higher concentration  encourages further self-assembly. 

\subsection{Zeta potentials}
The zeta potentials obtained for all four target compounds are positive. This shows that the lysine binding groups on the nanoscale interface are protonated at physiological pH. 
\newline
C\textsubscript{16}-L-Ala-L-Lys and C\textsubscript{16}-D-Ala-D-Lys having very similar \textzeta-potentials to each other (within error) while C\textsubscript{16}-L-Ala-D-Lys and C\textsubscript{16}-D-Ala-L-Lys have higher \textzeta-potentials of ca. 45 mV. If binding were to depend solely on charge density, it might be expected that C\textsubscript{16}-L-Ala-D-Lys and C\textsubscript{16}-D-Ala-L-Lys would perform more effectively.\textsuperscript{\cite{Chan2016ChiralBinding}} 
\newline
This increased \textzeta -potential for the C\textsubscript{16}-L-Ala-D-Lys and C\textsubscript{16}-D-Ala-L-Lys pair could explain their propensity towards further aggregation in comparison to C\textsubscript{16}-L-Ala-L-Lys and C\textsubscript{16}-D-Ala-D-Lys, and thus explain why larger aggregates are observed by DLS for the former pair of systems.

\subsection{Transmission Electron Microscopy Analysis}
Transmission Electron Microscopy was used to image the nanoscale binders alone, and in the presence of both biological polyanions of interest in this work. It can be shown by TEM that the nanoscale binders undergo self-assembly, and the micelles have a diameter of 5 $\pm$ 1 nm. This is smaller than the diameter observed via DLS, this occurs because of the differing conditions in which the micelles are sized. DLS measures the micelles in solution and therefore determines hydrodynamic diameter, whilst TEM samples are dried and hence only the nanoscale assembly itself is observed. 

\subsection{Circular Dichroism}
Circular Dichroism shows that the four target compounds synthesised in this work comprise of two enantiomeric pairs of molecules, which absorb plane polarised light in a broadly equal and opposite fashion. 
\newline
It has also been noted that there are significant differences between the CD spectra obtained for the diastereomeric pairs e.g. C\textsubscript{16}-L-Ala-L-Lys and C\textsubscript{16}-L-Ala-D-Lys, in terms of both absolute ellipticity and line shape. These differences in CD help support the hypothesis that these diastereomeric pairs fold themselves in very different ways and help explain why they possess differing propensities towards self-assembly. 
\section{Polyanion Binding Studies}
\subsection{Ethidium Bromide Displacement Assay}
The results from the Ethidium Bromide displacement assay show that all four novel nanostructures do bind DNA and successfully displace EthBr at micromolar concentrations. Again, this is much lower than the CAC determined by Nile Red and implies that the presence of DNA has an effect on binding the biological polyanion. CE\textsubscript{50} values show a dramatic preference for D-Lys as the binding group over L-Lys. This suggests that the binding interface with DNA is dominated by the precise interaction with the lysine binding group, which is determined by chirality. The chirality of the alanine spacer group, and the diastereomeric nature of the self-assembled systems appear to have no effect on the recognition interface. 

\subsection{Mallard Blue Assay}
Mallard Blue assays have shown that if the target compounds are unable to self-assemble, then their heparin binding ability is "switched off".  It can be stated that self-assembly  of these compounds is necessary before heparin binding and MalB displacement is possible. 

CE\textsubscript{50} values show that the C\textsubscript{16}-L-Ala-L-Lys and C\textsubscript{16}-D-Ala-D-Lys pair behave similarly to each other, and so too do C\textsubscript{16}-L-Ala-D-Lys and C\textsubscript{16}-D-Ala-L-Lys. Furthermore, the former pair of binders do appear to somewhat outperform the latter, and suggests that the nature of the self-assembled system is much more dominant for heparin binding. The EC\textsubscript{50} values obtained by Mallard Blue assay are different to the CACs. Therefore, it can be hypothesised that the presence of heparin encourages self-assembly. 

The chiral selectivity noted previously by Chan and Smith is still present, to some extent as it is only observed for C\textsubscript{16}-L-Ala-D-Lys and C\textsubscript{16}-D-Ala-L-Lys, but not the other two synthesised systems. This suggests that this chiral selectivity is dependent on the precise conformational orientation of the cationic surface ligands. 

\subsection{TEM Analysis of Polyanion Binding}
In the presence of DNA or heparin, hierarchical aggregation of the cationic micelles is observed by TEM. This hierarchical aggregation has been understood as containing close packed cationic micelles stabilised by the surrounding linear polyanions, as fully characterised by Vieira \textit{et al.}\textsuperscript{\cite{Vieira2017EmergenceHeparin}} The binder:DNA aggregates that are observed when the binder is imaged in the presence of DNA, are also small enough (< 150 nm in diameter) that they may successfully undergo endocytosis into mammalian cells and hence could potentially be useful DNA transfection agents.\textsuperscript{\cite{Ghosh2008EfficientNanoparticles}} 

This phenomenon also provides more detail as to why the CE\textsubscript{50} values indicate that more than one positive charge is necessary to bind one negative charge on the polyanion. These images show that not all of the positive charges on these nanoscale interfaces can be arranged so that they can bind directly to the target. 

In summary, it is apparent that different systems herein are optimised for heparin or DNA, revealing fundamental differences in the way these polyanions bind to cationic targets. In general, DNA appears to be more selective of the precise ligand it choses to interact with - a consequence of its rigid, well-defined structure. In contrast, heparin prefers the systems which has the greatest propensity towards self-assembly, reflecting its adaptive, polydisperse nature. However, for C\textsubscript{16}-L-Ala-D-Lys and C\textsubscript{16}-D-Ala-L-Lys, which self-assembly poorly, some chiral ligand preference is observed. 
