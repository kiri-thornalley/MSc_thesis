% Chapter Template

\chapter{Further Work} % Main chapter title
\label{Chapter4} % Change X to a consecutive number; for referencing this chapter elsewhere, use \ref{ChapterX}
Clearly there is much work to still be done with these self-assembling nanoscale binders, before any would have a chance of obtaining clinical approval.  

Firstly, as one of the objectives for this work was to develop a viable protamine sulfate mimic for clinical use and/or a non-viral gene delivery method for genetic therapy,  it is clear that the biocompatibility of these nanoscale assemblies would need to be determined. Previous work by Kim \textit{et al.} has shown that endogenous amino acids are more biologically compatible than their exogenous analogues.\textsuperscript{\cite{Kim2016PolycationsApplications}} By extention, this supposes that the C\textsubscript{16}-D-Ala-D-Lys system would be the least biologically compatible nanoscale system that has been synthesised in this work. This is unfortunate, as this system does bind particularly well to both heparin sulfate and DNA. However, the use of D-amino acids does offer the advantage of avoiding the breakdown by peptide enzymes which can only work on the L-form. 
\newline
The biocompatibility and cytotoxicity of these compounds can be assessed by the agar diffusion assay.\textsuperscript{\cite{InternationalStandardsOrganisation-ISO10993-5:2009Https://www.iso.org/standard/36406.html}} However, it is also known that this assay has several limitations, most importantly it shows only the short term (acute) effects of the compound. If a drug affects cell functionality without being cytotoxic, it will not be apparent in this assay.\textsuperscript{\cite{Pusnik2016TheCompounds}}

Secondly, if any of these nanoscale binders are biologically compatible, it would need to be seen if they still function in more competitive media. This would first involve assessing the binding of both heparin and DNA in human serum instead of the Tris-HCl buffer that has been used previously, before proceeding to using human plasma. It is in these more competitive conditions where the heparin binders synthesised by Bromfield \textit{et al.} previously failed.\textsuperscript{\cite{Bromfield2014NanoscaleMedia}}

Also, this system could be modified again to potentially tune the system even further towards heparin and/or DNA. It would be interesting to see the effect of modifying the alanine spacer group by replacing it with a different amino acid. However, there is the risk that this could be deleterous to the systems ability to self-assemble if the spacer group is too highly charged.\textsuperscript{\cite{Fechner2016ElectrostaticBinding}}  From previous results reported herein, it is unlikely to be wise to swap the alanine spacer group to an amino acid containing a phenyl ring, as this is likely to have a dramatic effect on the solubility of the molecule. Further modifications also include changing the hydrophobic component so that the CAC of the system is lowered, and hence the self-assembled system would be more stable. 

It has been shown in this work that the presence of either heparin sulfate or DNA dramatically lowers the EC\textsubscript{50} values to significantly below the CAC as determined by Nile Red. Does this occur in the presence of any other biological polyanions? 

Molecular dynamics calculations could be performed on the novel binders synthesised in this work to investigate whether the hypothesis proposed by Boutellier \textit{et al.}  to explain the difference in the shape of the self-assembled BTA-Val system on changing one chiral centre, also holds for the L/D-Ala-L/D-Lys systems here. 
These calculations could also explain why the systems synthesised here containing the alanine spacer group, show an enhanced affinity towards heparin sulfate and DNA, in comparison to those from Chan and Smith. As yet, it has only been hypothesised that they are "better able to express their chiral information at the self-assembled interface".\textsuperscript{\cite{Chan2016ChiralBinding}}

Finally, it would perhaps be useful to compare the DNA binding affinity of the novel compounds in this work to that of commercially available DNA transfection agents such as Lipofectamine,\footnote{This idea came from a discussion with Dr Christopher Serpell at the University of Kent, for which I am grateful.} as well as assessing whether these novel systems can indeed undergo endocytosis into mammalian cells. 
